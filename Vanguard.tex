\documentclass{adsdoc}

\newcommand{\doctitle}{Variante en EDH Commander : les rôles « Vanguard »}

\title{
    \vspace*{-2cm}
    Variante EDH Commander : les rôles « Vanguard »\\
    \Large{Ajoutez des rôles uniques à vos parties d'EDH Commander, pour plus de fun et de variété !}\\
    {
        \large
    \vspace*{0.2cm}
    -- \textbf{Nombre de joueurs :} 3 à 6 joueurs (idéalement 4) \\
    -- \textbf{Durée d'une partie :} comme une partie d'EDH standard, 1H à 2H \\
    -- \textbf{Matériel requis :} un petit paquet des 32 cartes « Vanguard » fourni, en plus de vos decks d'EDH préférés
    }
}
\date{}

\begin{document}

\maketitle
\thispagestyle{fancy}

\vspace*{-1.5cm}

\noindent

\begin{minipage}[t]{0.65\textwidth}
\vspace{0pt}
La variante \textbf{Vanguard} ajoute une dose d'originalité et une touche unique à vos parties de \textbf{Magic: The Gathering} au format EDH ou duel construit, en vous ajoutant un genre d'emblême unique et puissant.\\
Vous incarnez un personnage du lore de Magic des années 1996/1998.\\
Ce rôle confère des avantages et des inconvénients uniques, modifiant la façon dont le joueur interagit avec le jeu.
Notamment, cela modifie la taille initiale et la limite de main (ex. +3 ou -2) et les points de vie de début de partie (ex. -7 ou +18).\\
Si besoin (en anglais) : \url{mtg.wiki/page/Vanguard_(format)}\\
Ce format est présent dans les règles officielles de Magic (§ 902. Vanguard), mais n'est que très rarement joué en EDH !\\
Voici les règles de la variante « Vanguard » :
\end{minipage}\hfill
\begin{minipage}[t]{0.05\textwidth}
    \vspace{0pt}
\end{minipage}\hfill
\begin{minipage}[t]{0.25\textwidth}
\vspace{0pt}
\hfill
\includegraphics[width=\linewidth]{images/pvan-102-gerrard.png}
\end{minipage}

\subsection*{Distribution initiale et choix des rôles « Vanguard »}

\begin{itemize}
    \item Choisissez d'abord le niveau de votre table, puis chaque personne choisit son deck, comme dans une partie standard.
    \item \textbf{Distribution initiale des cartes ``Vanguard''} : chaque joueur reçoit trois cartes Vanguard, aléatoirement choisies parmi les 32 disponibles.
    \item \textbf{Choix équilibré} : le but est ensuite de choisir un rôle qui sera utile à votre stratégie et ne l'annulera pas, mais il faut éviter de choisir le rôle ``ultra puissant'' qui écraserait le reste de la table.
    \begin{itemize}
        \item
        Exemple de choix "trop fort" : \emph{Titania} autorise à poser deux terrains par tour, mais si votre deck repose sur le « Toucheterre », cela sera trop fort.
        \item
        Exemple de choix "trop faible" : \emph{Karn} empêche les équipements de s'attacher à des créatures, donc cela peut neutraliser un deck voltron équipement.
    \end{itemize}
    \item
    Il faut donc choisir le meilleur rôle, parmi les trois rôles reçus. Les autres cartes non sélectionnées sont replacées dans le paquet, et peuvent être rapportées avant le début de la partie à Lilian\footnote{N'oubliez pas non plus de rapporter les cartes employées, à la fin de la partie !}.
\end{itemize}

\subsection*{Règles des rôles « Vanguard »}

\begin{itemize}
    \item Chaque carte « Vanguard » modifie les points de vie initiaux et la taille de la main initiale, ainsi que des capacités spéciales. Par exemple, ``Urza'' donne une capacité activée qui dit « 3 : Urza inflige 1 blessure à n'importe quelle cible ».\\
    \item Les cartes « Vanguard » sont placées face visible devant chaque joueur, et leurs effets sont actifs en tout temps. Révélez chacun vos cartes « Vanguard » en même temps, avant de commencer la partie.
    \item Les cartes « Vanguard » ne font pas partie du deck du joueur, et ne peuvent pas être affectées par des sorts ou des capacités.
    \item Les joueurs ne peuvent pas échanger leurs cartes « Vanguard » entre eux avant le début de la partie.
\end{itemize}

\subsection*{Exemples de cartes « Vanguard »}

\hfill{}

\begin{center}
    % \includegraphics[width=0.29\linewidth]{images/pvan-102-gerrard.png}
    \includegraphics[width=0.29\linewidth]{images/pvan-103-karn.jpg}
    \includegraphics[width=0.29\linewidth]{images/pvan-406-titania.jpg}
    \includegraphics[width=0.29\linewidth]{images/pvan-407-urza.jpg}
\end{center}

\hfill{}

\subsection*{Avec d'autres variantes ?}

Ce mode de jeu « Vanguard » est compatible avec d'autres variantes, comme « Shogun », « Treachery » ou « Planechase » !

Nous avons déjà testé une partie à quatre joueurs avec « Shogun », « Planechase » \textbf{et} « Vanguard », en même temps.
Cela marche très bien, il faut juste être bien concentré !

\hfill{}

\end{document}
