\documentclass{adsdoc}

\newcommand{\doctitle}{Règles format fun : « Judge Tower »}

\title{
    \vspace*{-2cm}
    La « \textbf{Judge Tower} »\\
    \Large{Préparez-vous à une bataille épique qui mettra à dure épreuve vos connaissances des règles à \textbf{Magic: the Gathering}}\\
    {
        \large
    \vspace*{0.2cm}
    -- \textbf{Nombre de joueurs :} 2 à 6 joueurs (idéalement 3-4) \\
    -- \textbf{Durée d'une partie :} environ 30 minutes à 1 heure \\
    -- \textbf{Matériel requis :} un énorme deck fourni, besoin de D6 / D20
    }
}
\date{}

\begin{document}

\maketitle
\thispagestyle{fancy}

\vspace*{-1.5cm}

\noindent

\begin{minipage}[t]{0.65\textwidth}
\vspace{0pt}
La \textbf{Judge Tower} est un format \textbf{Magic: The Gathering} occasionnel, idéal pour la détente et l'entraînement intensif sur la complexité des règles.\\
Ce format exigeant met l'accent sur la connaissance des règles et la prise de décision sous pression.\\
Chaque joueur doit naviguer à travers un ensemble unique de règles qui transforment chaque partie en un défi stratégique.\\
Si besoin (en anglais) : \url{mtg.wiki/page/Judge_Tower}\\
Voici les règles de la Judge Tower, conçues pour maximiser la difficulté et l'apprentissage.
\end{minipage}\hfill
\begin{minipage}[t]{0.05\textwidth}
    \vspace{0pt}
\end{minipage}\hfill
\begin{minipage}[t]{0.25\textwidth}
\vspace{0pt}
\hfill
\includegraphics[width=\linewidth]{images/rules_lawyer.jpg}
\end{minipage}

\subsection*{Principes Fondamentaux et Conditions de Jeu}

\begin{itemize}
    \item \textbf{Objectif : ``Je Peux $\Rightarrow$ Je Dois''} : Dès qu'une action (jouer un sort, activer une capacité, etc) est légalement possible, elle \textbf{doit} être effectuée. Toutes les options de type \emph{« you may »} deviennent obligatoires, si elles peuvent être remplies.
    \item \textbf{Ressources Infinies} :
    \begin{itemize}
        \item Tous les joueurs ont un total de points de vie \textbf{infini}.
        \item Tous les joueurs ont un \textbf{mana infini} de n'importe quel type (toutes les couleurs, incolore, neigeux), disponible sans passer par la pile.
    \end{itemize}
    \item \textbf{Partage du Deck et du Cimetière} : Tous les joueurs partagent le même deck (un très grand nombre de cartes uniques) et le même cimetière.
    \item \textbf{Mains Révélées} : Les joueurs jouent avec leurs mains \textbf{révélées}. Tout ce qui devrait être caché est révélé, sauf si une carte impose explicitement de choisir une pile face cachée (ex. Rançon de Sauron).
    \item \textbf{Propriété des Cartes} : Chaque joueur conserve la propriété de toutes les cartes qu'il pioche. Il est crucial de se souvenir de quelles cartes ont été piochées par qui.
    \item \textbf{Main de Départ} : Chaque joueur commence avec \textbf{trois (3) cartes} en main, au lieu de zéro.
    Attention : un nombre plus élevé rend les erreurs quasi inévitables dès le premier tour.
    \item \textbf{Coût de Mana Variable (X)} : Si un sort ou une capacité a un coût de mana de $X$, $X$ est toujours égal à $\mathbf{3}$, sauf si jouer avec $X=3$ est illégal. Dans ce cas, la plus petite valeur légale possible doit être choisie.
    \item \textbf{Annoncer ses phases} : le joueur actif annonce oralement ses phases et étapes, pour clarifier le déroulement du tour.
\end{itemize}

\newpage

\subsection*{Obligations d'Action}

\begin{itemize}
    \item \textbf{Jouer les Sorts} : Vous devez jouer toutes les cartes de votre main, dès que cela est légalement possible.
    \item \textbf{Capacités Activées} : Vous devez activer chaque capacité activée des permanents que vous contrôlez (et des cartes dans le cimetière, le dessus de la bibliothèque ou l'exil), une fois par tour (s'il existe une cible légale), dès que cela est légalement possible. Activez toujours les capacités du bas de la carte vers le haut.
    \item \textbf{Combat} : Vous devez attaquer avec tous les attaquants légaux et bloquer avec tous les bloqueurs légaux, chaque fois que l'option est présentée.
    \item \textbf{Modes Optionnels} : Tous les modes optionnels (ex. « Choisissez l'un des modes suivants : ») sont obligatoires dans la mesure où vous pouvez les remplir.
\end{itemize}

\subsection*{Système de Sanction (Variante des Points de Vie de Règle)}

\begin{itemize}
    \item \textbf{Niveau de Compétence} : Au début de la partie, chacun estime son niveau de compétence sur un D6 (de $1 =$ confiant, à $6 =$ relax). Ce sont vos « points de vie de règle ».
    \item \textbf{Erreurs (Violations de Règle)} : Chaque fois qu'un joueur commet une violation des règles de Magic ou des règles spécifiques de la Judge Tower (et que l'erreur est remarquée par n'importe quel joueur), le joueur fautif perd $\mathbf{1}$ point de vie de règle.
    \item \textbf{Fin de Partie} : La partie se termine lorsqu'il y a un gagnant (PV de règles des adversaires tombés à zéro). Le gagnant marque un point sur une fiche de score. Les cartes en jeu, en main, et au cimetière sont exilées, et une nouvelle partie peut commencer.\\
    Le premier à atteindre trois victoires est déclaré \textbf{Grand Maître de la Judge Tower} !
\end{itemize}

%\begin{center}
%    \includegraphics[width=1.5cm]{images/command-tower.jpg}
%\end{center}

\subsection*{Contenu du Deck (La Tour de Cartes)}

Le deck est divisé en deux boîtes (une difficile à la boîte dorée, l'autre plus « fun » à la boîte argentée), et est spécifiquement conçu pour mettre les règles à l'épreuve.

\begin{itemize}
    \item \textbf{Cartes Unset} : Quelques cartes à bords argentés avec des règles insolites (ex. \emph{Do-It-Yourself Seraph}).
    \item \textbf{Cartes Obscures/Erratées} : Cartes très anciennes, ou très récentes, avec des mécaniques désuètes (\emph{Banding}, \emph{Phasing}), des formulations bizarres ou fausses (\emph{Bloodvial Purveyor}), ou ayant reçu des erratas cruciaux. Trouvez\footnote{Sortez un téléphone et consultez le Texte Oracle, sur \url{Scryfall.com} ou \url{Gatherer.Wizards.com} !} la règle correcte !
    \item \textbf{Exclusions Volontaires} :
    \begin{itemize}
        \item Très peu de cartes produisant des jetons non-sacrifiables (sauf \emph{Food}, \emph{Treasure}, \emph{Clue}, etc.) pour éviter la gestion des marqueurs.
        \item Peu de \emph{Planeswalkers}\footnote{et Lilian n'aime pas ces cartes là !}, car ils sont trop vite gérés et demandent des dés.
        \item Aucune carte ``\emph{sharpied}'' ou provenant d'autres jeux. Seules de vraies cartes \textbf{Magic: The Gathering} sont autorisées. La ``Party Box'' propose cela par contre !
    \end{itemize}
\end{itemize}

\end{document}
