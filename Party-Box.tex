\documentclass{adsdoc}

\newcommand{\doctitle}{Règles format fun : « Party Box »}
\title{
    \vspace*{-2cm}
    La « \textbf{Party Box} »\\
    \Large{Essayez une session débile et amusante, à la croisée entre \textbf{Magic: the Gathering} et une douzaine d'autres jeux de cartes}\\
    {
        \large
    \vspace*{0.2cm}
    -- \textbf{Nombre de joueurs :} 3 à 5 joueurs (idéalement 4) \\
    -- \textbf{Durée d'une partie :} entre 1H et 1H30 \\
    -- \textbf{Matériel requis :} un énorme deck en commun, fourni, besoin de D6 / D20
    }
}
\date{}

\begin{document}

\maketitle
\thispagestyle{fancy}

\vspace*{-1.5cm}

\noindent

\begin{minipage}[t]{0.65\textwidth}
\vspace{0pt}
La \textbf{Party Box} est un format \textbf{Magic: The Gathering} occasionnel, idéal pour la détente et découvrir des cartes venant d'autres jeux TCG.\\
Ce format au haut potentiel de débilités met l'accent sur la libre interprétation de cartes venant d'autres TCG, dans une partie (quasi) ``normale'' de Magic en multi-joueurs.\\
Si besoin (en anglais) :
\url{TappedOut.net/mtg-decks/banned-restricted-and-fun-party-box/}\\
Voici les règles de ma Party Box, conçues pour maximiser la difficulté et l'apprentissage.
\end{minipage}\hfill
\begin{minipage}[t]{0.05\textwidth}
    \vspace{0pt}
\end{minipage}\hfill
\begin{minipage}[t]{0.25\textwidth}
\vspace{0pt}
\hfill
\includegraphics[width=\linewidth]{images/surprise-party.jpg}
\end{minipage}

\subsection*{Principes Fondamentaux et Conditions de Jeu}

Chaque « Party Box » a ses propres règles personnalisées, et les miennes sont les suivantes :

\begin{enumerate}
    \item Les joueurs utilisent une bibliothèque, un cimetière et une zone d'exil communs.
    \item Les joueurs n'ont pas de commandant.
    \item Chaque joueur commence avec 20 points de vie.
    \item Chacun peut attaquer qui il veut, comme en EDH Commander.
    \item La politique et les discussions sont au cœur de cette variante, comme en EDH !
    \item Chaque carte de votre main peut être jouée face visible (en respectant ses conditions), OU bien être placée face cachée comme un terrain ``arc-en-ciel'' de base (un "Partout"), qui peut produire n'importe quel type de mana.
    \item Le deck contient un certain nombre de cartes ``joker'' (un dos de carte, Magic ou d'un autre jeu). Une carte joker PEUT ÊTRE une carte quelconque du jeu d'où elle vient (ex. un dos de Magic peut être un "Black Lotus", ou un "Ancestral Recall"), mais un joker NE PEUT PAS être une carte jouée lors de cette session. Cela inclut les cartes réelles, ou les cartes sélectionnées avec des cartes joker précédentes. Tous les coûts associés au lancement doivent toujours être payés. Il faut évidemment connaître par cœur la carte choisie !
    \item Si une carte doit être placée au bas du paquet ou mélangée dans le paquet, elle est à la place exilée. \emph{Nous sommes ici pour jouer, pas pour mélanger un deck gigantesque.}

\newpage

    \item Il y a des cartes Magic ``sharpied'', c'est-à-dire modifiées à la main avec un marqueur noir permanent, pour changer leur texte, leurs coûts, leurs forces/endurance, etc. Ces cartes sont légales et doivent être jouées selon leur texte modifié.

    \item Chaque joueur commence la partie avec en main une copie de ``Voloscille'' (``Flickerwisp''), qui peut être jouée normalement, une fois dans la partie. Cela permet de gérer les cartes problématiques (un Pokémon gigantesque ?), mal interprétées, ou de faire un play assez fun : posez une carte incroyable mais qui n'a pas de coup de mana (un Pokémon gigantesque !!) face cachée, comme un terrain ``arc-en-ciel'', puis ``blinkez'' ce permanent ! Il reviendra face visible, en fin de tour !

    \item Les effets des tuteurs ont leurs propres résultats spécifiques : en une phrase, le joueur ne choisira pas sa carte, mais il va "cascader" jusqu'à trouver la première carte légale selon les critères du tuteur. Plus précisément :
    \begin{enumerate}
        \item Tuteurs pouvant rechercher N'IMPORTE QUELLE carte (vampirique, démoniaque, etc.) : lors du lancement, vous pouvez déclarer la VALEUR DE MANAS de la carte que vous souhaitez. Exilez les cartes face visible du dessus de votre bibliothèque jusqu'à ce que vous révéliez une carte avec cette valeur de manas. Mettez cette carte dans votre main. (les autres cartes restent exilées. La valeur de mana DOIT être inférieure ou égale à 8 pour éviter le deckout).
        \item Tuteurs qui spécifient le type de carte à piocher (Éclairé, Enterré, etc.) : exilez les cartes face visible du dessus de votre bibliothèque jusqu'à ce que vous révéliez une carte du type spécifié. Mettez cette carte dans votre main.
    \end{enumerate}
\end{enumerate}

\subsection*{Des cartes venant d'autres TCG : Pokémon, Yu-Gi-Oh! etc}

Dans cette Party Box, il y a des cartes provenant d'autres jeux de cartes à collectionner (TCG) comme Yu-Gi-Oh!, Pokémon, Flesh and Blood, etc.
Lorsqu'une de ces cartes est piochée, elle est interprétée comme une carte Magic.
C'est là tout le plaisir de la Party Box, il faut faire au mieux !

Pour les cartes qui seraient interprétées comme des créatures, si leur force ou leur endurance termine par des zéros (ex. 60 pour un Pokémon, 1500 pour un Yu-Gi-Oh! ou un One Piece), retirez les zéros en fin du nombre (ex. 6 ou 15).

Pour toutes les cartes qui seraient interprétées autrement que comme des créatures, il faut faire de son mieux pour les jouer ``comme si c'était des cartes Magic''.

Soyez créatifs, discutez-en entre vous, et en cas de doute, venez demander à Lilian !

\hfill{}

\begin{center}
    % \includegraphics[width=0.19\linewidth]{images/mise.jpg}
    \includegraphics[width=0.24\linewidth]{images/party-box-exemple-world-of-warcraft.png}
    \includegraphics[width=0.24\linewidth]{images/party-box-exemple-pokemon.png}
    \includegraphics[width=0.24\linewidth]{images/party-box-exemple-lanfeust.png}
    \includegraphics[width=0.24\linewidth]{images/party-box-exemple-harry-potter.png}
\end{center}

% \hfill{}
%
% \newpage

\subsection*{Une variante supplémentaire à la ``Party Box'' : les cartes ``Pictionary'' !}

Si vous voulez pimenter\footnote{Cette variante a été mise au point par une discussion entre Lilian et Mathieu de la boutique Vent Divin, au LTC en avril 2025, et a été testée plusieurs fois. Cependant, c'est hautement ``role-play'', donc à utiliser avec modération et seulement si vous êtes à l'aise.} encore plus le jeu, vous pouvez utiliser les cartes ``Pictionary'' présentes dans la boîte.

\begin{itemize}
    \item Mélangez au début de parties ces cartes, et placez-les en une pile face cachée, au milieu de la table.
    \item Quand un sort non-incolore est en pile, un (seul) joueur peut décider de piocher la carte Pictionary du dessus du paquet, et la révéler.
    \item Selon les couleurs du sort en pile, cela donne entre un et cinq mots de la carte Pictionary (jaune = blanc, bleu, marron = noir, vert, rouge).
    \item Le joueur qui a révélé la carte Pictionary doit alors mettre au point une règle spéciale pour cette partie, en lien avec les mots ainsi sélectionnés. Visez des petites règles amusantes, des mini-jeux, des légères variations, mais rien de trop fort. Il ne faut pas que cela ``casse'' l'équilibre de la partie
    \item Si cette nouvelle ``mini-règle'' est assez drôle et originale, et que le reste de la table l'autorise (vous pouvez en discuter), elle est désormais valide.
\end{itemize}

\begin{center}
    \includegraphics[width=0.650\linewidth]{images/pictionnary.jpg}
\end{center}

Exemple réellement employé lors d'une partie en mai 2025 (à six joueurs) : le sort avait donné les trois mots "Bûcheron", "Neige", et "Cave". Comme ce mélange faisait fortement penser à Noël, j'avais choisi la règle additionnelle suivante : ``si un sort lancé plus tard fait unanimement penser à Noël, alors tout le monde reçoit le cadeau de gagner la partie, qui s'arrêtera alors''.
Après quasiment 1h30 de jeu délirant et sans interruption, quelqu'un a lancé \emph{je ne sais plus quel sort}, qui nous a fait penser à Noël, et alors nous étions contents de pouvoir conclure en gagnant tous ensemble !

\end{document}
