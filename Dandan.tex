\documentclass{adsdoc}

\newcommand{\doctitle}{Règles format fun : « Dândan »}

\title{
    \vspace*{-2cm}
    Le format « \textbf{Dândan} »\\
    \Large{Découvrez un format unique et accessible, idéal pour les joueurs de tous niveaux à \textbf{Magic: the Gathering}}\\
    {
        \large
    \vspace*{0.2cm}
    -- \textbf{Nombre de joueurs :} 2 joueurs (format duel) \\
    -- \textbf{Durée d'une partie :} environ 10 à 20 minutes \\
    -- \textbf{Matériel requis :} deck de 5 cartes uniquement, des terrains de base
    }
}
\date{}

\begin{document}

\maketitle
\thispagestyle{fancy}

\vspace*{-1.5cm}

\noindent

\begin{minipage}[t]{0.65\textwidth}
\vspace{0pt}
Le format \textbf{Dândan} (aussi appelé « Forgetful Fish » en anglais) est un format \textbf{Magic: The Gathering} occasionnel, ultra-accessible et rapide.\\
Ce format minimaliste met l'accent sur la simplicité et l'accessibilité, tout en offrant des parties stratégiques et amusantes.\\
Le nom vient de la carte \emph{Dândân}, une créature légendaire qui ne peut pas attaquer à moins que le joueur adverse ne contrôle une île.\\
Si besoin (en anglais) : \url{mtg.wiki/page/Forgetful_Fish}\\
Voici les règles du format Dândan, conçues pour des parties rapides et amusantes.
\end{minipage}\hfill
\begin{minipage}[t]{0.05\textwidth}
    \vspace{0pt}
\end{minipage}\hfill
\begin{minipage}[t]{0.25\textwidth}
\vspace{0pt}
\hfill
\includegraphics[width=\linewidth]{images/command-tower.jpg}
\end{minipage}

\subsection*{Principes Fondamentaux et Construction de Deck}

\begin{itemize}
    \item \textbf{Taille du Deck} : Chaque joueur construit un deck de \textbf{exactement 5 cartes} (hors terrains de base).
    \item \textbf{Terrains de Base} : En plus de ces 5 cartes, chaque joueur a accès à une réserve infinie de terrains de base (Plaine, Île, Marais, Montagne, Forêt).
    \item \textbf{Restrictions de Construction} :
    \begin{itemize}
        \item Les 5 cartes doivent être toutes différentes (pas de doublons).
        \item Aucune carte rare ou mythique n'est autorisée (uniquement communes et peu communes).
        \item Les cartes doivent être légales en format Eternal (pas de cartes bannies, de cartes ``Un-'' à bords argentés).
    \end{itemize}
    \item \textbf{Simplicité} : Ce format est idéal pour les nouveaux joueurs ou pour des parties rapides entre amis, sans avoir besoin de construire des decks complexes ou coûteux.
\end{itemize}

\subsection*{Règles de Jeu}

\begin{itemize}
    \item \textbf{Points de Vie} : Chaque joueur commence avec \textbf{20 points de vie}.
    \item \textbf{Main de Départ} : Au début de la partie, chaque joueur pioche \textbf{3 cartes} de son deck de 5 cartes.
    \item \textbf{Mulligan Simplifié} : Si un joueur n'est pas satisfait de sa main de départ, il peut mélanger et piocher une nouvelle main de 3 cartes. Il peut répéter cette opération autant de fois que souhaité, mais doit garder la dernière main piochée.
    \item \textbf{Terrains de Base} : À chaque tour, au lieu de piocher une carte, un joueur peut choisir de \textbf{piocher une carte de son deck OU placer un terrain de base de son choix} directement sur le champ de bataille, engagé.
    \begin{itemize}
        \item Cette règle spéciale remplace la pioche normale au début du tour (sauf au premier tour du premier joueur, où il n'y a pas de pioche).
        \item Un joueur peut toujours poser un terrain pendant sa phase principale, selon les règles normales de Magic.
    \end{itemize}
    \item \textbf{Deck Vide} : Si un joueur doit piocher une carte alors que son deck est vide, il ne perd pas la partie. Il continue simplement à jouer sans piocher.
    \item \textbf{Conditions de Victoire} : La partie se termine normalement quand un joueur atteint 0 points de vie ou moins, ou par d'autres conditions de victoire/défaite standard de Magic.
\end{itemize}

\newpage

\subsection*{Stratégies et Conseils}

\begin{itemize}
    \item \textbf{Équilibre du Deck} : Avec seulement 5 cartes, chaque choix compte ! Essayez de trouver un bon équilibre entre créatures, sorts et enchantements/artefacts.
    \item \textbf{Courbe de Mana} : Pensez à votre courbe de mana. Il est généralement conseillé d'avoir des cartes de différents coûts de mana (par exemple : 1, 2, 3, 4, et 5).
    \item \textbf{Synergie} : Même avec 5 cartes, vous pouvez créer des synergies intéressantes. Par exemple, des cartes qui se renforcent mutuellement ou qui fonctionnent bien ensemble.
    \item \textbf{Polyvalence} : Les cartes polyvalentes (qui peuvent servir en attaque et en défense, ou qui ont plusieurs modes) sont particulièrement précieuses dans ce format.
    \item \textbf{Expérimentation} : N'hésitez pas à tester différentes combinaisons de cartes. Le format Dândan est parfait pour expérimenter rapidement de nouvelles idées !
\end{itemize}

\subsection*{Exemples de Decks Dândan}

\textbf{Deck Aggro Rouge :}
\begin{enumerate}
    \item \emph{Lightning Bolt} (sort d'éphémère, 1 mana rouge)
    \item \emph{Shock} (sort d'éphémère, 1 mana rouge)
    \item \emph{Goblin Guide} (créature 2/2, 1 mana rouge)
    \item \emph{Keldon Marauders} (créature 3/3, 2 manas rouges)
    \item \emph{Ball Lightning} (créature 6/1, 3 manas rouges)
\end{enumerate}

\textbf{Deck Contrôle Bleu :}
\begin{enumerate}
    \item \emph{Counterspell} (sort d'éphémère, 2 manas bleus)
    \item \emph{Mana Leak} (sort d'éphémère, 2 manas dont 1 bleu)
    \item \emph{Unsummon} (sort d'éphémère, 1 mana bleu)
    \item \emph{Cloud of Faeries} (créature 1/1, 2 manas bleus)
    \item \emph{Man-o'-War} (créature 2/2, 3 manas dont 2 bleus)
\end{enumerate}

\subsection*{Variantes Possibles}

Pour varier les plaisirs, vous pouvez essayer ces variantes au format Dândan :

\begin{itemize}
    \item \textbf{Dândan Construit} : Autorisez les cartes rares et mythiques, pour des decks plus puissants.
    \item \textbf{Dândan Limité} : Chaque joueur reçoit un booster aléatoire et doit construire son deck de 5 cartes à partir des cartes du booster.
    \item \textbf{Dândan Multi-joueurs} : Jouez à 3 ou 4 joueurs, en adaptant les règles (par exemple, attaquer n'importe quel adversaire).
    \item \textbf{Mini-Tournoi Dândan} : Organisez un petit tournoi rapide avec plusieurs manches, où le gagnant est celui qui remporte le plus de parties.
\end{itemize}

\subsection*{Pourquoi Jouer au Dândan ?}

\begin{itemize}
    \item \textbf{Accessibilité} : Parfait pour les débutants ou pour initier de nouveaux joueurs à Magic.
    \item \textbf{Rapidité} : Les parties sont très rapides, idéales pour combler un temps d'attente.
    \item \textbf{Créativité} : Avec seulement 5 cartes, chaque deck est unique et reflète votre style de jeu.
    \item \textbf{Économique} : Pas besoin d'investir dans des cartes coûteuses pour jouer et s'amuser !
    \item \textbf{Rejouabilité} : Changez facilement une ou deux cartes pour tester de nouvelles stratégies.
\end{itemize}

\end{document}
