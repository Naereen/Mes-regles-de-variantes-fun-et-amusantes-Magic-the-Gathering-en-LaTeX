\documentclass{adsdoc}

\newcommand{\doctitle}{Règles format fun : le « Dandân »}

\title{
    \vspace*{-2cm}
    Le « \textbf{Dandân} », ou \emph{Forgetful Fish}\\
    \Large{un duel intense et stratégique avec un deck partagé, n'ayant qu'une seule condition de victoire}\\
    {
        \large
    \vspace*{0.2cm}
    -- \textbf{Nombre de joueurs :} 2 uniquement\\
    -- \textbf{Durée d'une partie :} 30 minutes à 45 minutes environ \\
    -- \textbf{Matériel requis :} un deck de 80 cartes est fourni (en anglais et français), besoin de D6 / D20
    }
}
\date{}

\begin{document}

\maketitle
\thispagestyle{fancy}

\vspace*{-1.5cm}

\noindent

\hfill
\begin{minipage}[t]{0.21\textwidth}
\vspace{0pt}
\hfill
\includegraphics[width=\linewidth]{images/sld-2138-dandan.jpg}
\end{minipage}\hfill
\begin{minipage}[t]{0.03\textwidth}
    \vspace{0pt}
\end{minipage}\hfill
\begin{minipage}[t]{0.50\textwidth}
\vspace{0pt}
La \textbf{Dandân} est un format \textbf{Magic: The Gathering} occasionnel, qui plaira aux joueurs passionnés des decks « contrôle » et saura faire réfléchir n'importe qui, des experts aux plus débutants.\\
Ce format occasionnel construit s'articule autour de la créature Dandân éponyme, une créature originaire de l'antique extension de « Arabian Nights ».\\
Voici les règles du format Dandân.
\end{minipage}\hfill
\begin{minipage}[t]{0.03\textwidth}
    \vspace{0pt}
\end{minipage}\hfill
\begin{minipage}[t]{0.21\textwidth}
\vspace{0pt}
\hfill
\includegraphics[width=\linewidth]{images/sld-2139-dandan.jpg}
\end{minipage}


\subsection*{Histoire du format et principe général}

Contrairement à la plupart des formats, les deux joueurs partagent le même deck de base (mono-bleu), et aussi le même cimetière.
Le Dandân a été créé par Nick Floyd entre le printemps 1996 et l'été 1997, mais il n'est devenu suffisamment populaire pour faire l'objet de quelques articles en ligne qu'en 2022 et a attiré l'attention lorsque la chaîne YouTube « Rhystic Studies » a mis en avant le format en 2023.

Wizards of the Coast a développé en 2025 un produit spécialement pour ce format, \emph{Secret Lair Chaos Vault: Dandân}.

Cette variante ressemble à un jeu de société basé sur les règles de \textbf{Magic}, car les joueurs n'apportent pas leurs propres decks.
Au lieu de cela, il n'y a qu'un seul deck commun, partagé entre les joueurs.

Le Dandân comprend un deck de 80 cartes.
Les listes peuvent varier, et la mienne n'est pas tout à fait celle que l'on peut retrouver en ligne ou dans le Secret Lair.


\subsection*{Le deck du Dandân}

Le format s'appelle aussi « Forgetful Fish » en anglais.
La partie « Forgetful » (oubliant) vient des huit exemplaires de «  Trou de mémoire » (Memory Lapse), tandis que la partie « Fish » (poisson) vient des dix exemplaires de Dandân.
Le reste du deck comprend une série de sorts bleus (éphémères et rituels, parfois des enchantements), qui servent à tirer parti du dessus du deck, de la pile et du poisson.

Le seul effet qui peut infliger des dégâts au joueur adversaire, dans ce format, est une attaque non bloquée d'un ou plusieurs Dandân.


\subsection*{Conditions de victoire}

Le Dandân se joue comme n'importe quelle autre partie de Magic. Les joueurs commencent avec 20 points de vie, ils piochent sept cartes, ils défaussent les cartes en surplus à la fin du tour et ils peuvent perdre la partie s'ils piochent dans un paquet vide.

\begin{center}
    \includegraphics[height=5cm]{images/arn-12-dandan.jpg}
    \includegraphics[height=5cm]{images/5ed-79-dandan.jpg}
    \includegraphics[height=5cm]{images/tsb-12-dandan.jpg}
\end{center}


\subsection*{Mains de départ et Mulligans}

Les mains de départ des joueurs sont distribuées comme au poker, une carte pour le premier joueur, une carte pour le deuxième joueur, et ainsi de suite. Si un effet oblige les deux joueurs à piocher une carte, le joueur actif distribue les cartes de la même manière que pour les mains de départ, en commençant par lui-même. Les deux joueurs partagent également un cimetière. Si un effet renvoie une carte de « votre » cimetière ou de votre bibliothèque, il fait désormais référence au cimetière partagé ou à la bibliothèque partagée. De plus, le « propriétaire » d'une carte est son lanceur.

En ce qui concerne les mulligans, les joueurs peuvent d'abord prendre un ou plusieurs mulligans gratuits pour les mains contenant moins de deux terrains ou sorts, en révélant toute main ne répondant pas à ces critères ; ensuite, une fois que les critères précédents ont été remplis, ils peuvent choisir de prendre d'autres mulligans pour lesquels les règles standard s'appliquent.


\subsection*{Plus d'informations si besoin ?}

N'hésitez pas à acheter le Secret Lair Chaos Vault: Dandân (lol), ou à vous imprimer votre propre version en proxies !
Plus d'information sur \url{https://mtg.wiki}

Enfin, je recommande les meilleures vidéos sur ce format, des chaînes YouTube « Rysthic Studies », « Tolarian Community College » et « CardMarket » :

\begin{center}
\includegraphics[height=2cm]{images/the-story-of-dandan-rysthic-studies-youtube.png}
% \href{https://www.youtube.com/watch?v=Otdgj8fBQxc}{Otdgj8fBQxc}
%
\includegraphics[height=2cm]{images/shuffle-up-and-play-tolarian-community-college-dandan.png}
% \href{https://www.youtube.com/watch?v=l9Oamtzm6E0}{l9Oamtzm6E0}
%
\includegraphics[height=2cm]{images/cardmarket-dandan.png}
% \href{https://www.youtube.com/watch?v=nBWXrXrcCLU}{nBWXrXrcCLU}
%
\end{center}

% \hfill{}

PS : si vous avez testé ce format et qu'il vous a intéressé, que diriez-vous de tester d'autres decks similaires, dans d'autres couleurs, ou basés sur d'autres concepts ?
Je pourrais imprimer d'autres decks pour la prochaine édition de la Fumble Corp. en 2026 !

% \href{https://www.youtube.com/watch?v=PUb8ygzu1Is}{YouTube: PUb8ygzu1Is}

\end{document}
