\documentclass{adsdoc}

\newcommand{\doctitle}{Variante en EDH Commander : « la Horde »}

\title{
    \vspace*{-2cm}
    Variante EDH Commander : « la Horde »\\
    \Large{Affrontez un ennemi surpuissant ensemble}\\
    {
        \large
    \vspace*{0.2cm}
    -- \textbf{Nombre de joueurs :} 1 à 5 joueurs (idéalement 3) \\
    -- \textbf{Durée d'une partie :} 1H à 2H \\
    -- \textbf{Matériel requis :} un deck d'une « Horde » fourni, en plus de vos decks d'EDH préférés
    }
}
\date{}

\begin{document}

\maketitle
\thispagestyle{fancy}

\vspace*{-1.5cm}

\noindent

La variante \textbf{la Horde} est conçue pour transformer une partie de \textbf{Magic: The Gathering} au format EDH en une partie « plusieurs contre une IA » (pilotée automatiquement par l'équipe).

Cette variante originale et amusante, qui rappellera peut-être les « decks de Raids » du TCG « World of Warcraft », change radicalement la manière de jouer en EDH, en vous proposant de vous unir à 1/2/3/4 joueurs et d'affronter une Horde quasi infinie d'ennemis, qui seront pilotés automatiquement.

Imaginez un peu, des vagues successives d'ennemis, avec des boss redoutables, que vous devrez vaincre en équipe ! Des Zombies, des Slivoïdes, ou d'autres !
%
Ce format n'est présent pas officiel, mais dispose d'un site dédié : \href{https://hordemagic.com/}{HordeMagic.com},
ainsi que d'un site dédié pour jouer en ligne si vous n'avez pas de deck spécifique pour une horde : \href{https://against-the-horde.com/}{Against-The-Horde.com}.
Sept decks de « la Horde » sont disponibles gratuitement en ligne, et vous pouvez aussi créer vos propres decks personnalisés.

\begin{center}
    \includegraphics[height=4cm]{images/la-horde.jpeg}
\end{center}

Voici les règles de la variante « la Horde ».

\subsubsection*{Principe fondamental}

Il s'agit d'un \textbf{mode coopératif}, de préférence pour \textbf{3 joueurs contre la Horde}.
La difficulté peut varier selon les Hordes.

Le deck de la Horde est composé d'environ \textbf{100-150 cartes}, pour $2/3$ de jetons et $1/3$ de vraies cartes (rituels, éphémères ou permanents).

\subsubsection*{Début de Partie}

Chaque joueur commence avec des points de vie et une taille de deck spécifiques, selon le nombre de joueurs (voir tableau plus bas).

Chaque joueur pioche sa main de départ normalement (7 cartes).

Les joueurs ont \textbf{3 tours} pour préparer leurs défenses et \textbf{ne peuvent pas attaquer} la Horde.

\subsubsection*{Emblème et Règles Spécifiques de la Horde}

Après le $3^{ème}$ tour des joueurs, la Horde commence son tour avec les règles suivantes :

\begin{itemize}
    \item \textbf{Emblème de la Horde} : « Les créatures que vous contrôlez ont la \textbf{célérité}, attaquent à chaque tour et \textbf{ne peuvent pas bloquer}. Ces créatures \textbf{ne peuvent pas attaquer les planeswalkers}. »
    \item La Horde a une \textbf{réserve de mana illimitée} pour payer ses coûts (de tout types de mana, y compris l'incolore et le neigeux).
    \item Les cartes renvoyées dans la main de la Horde passent \textbf{hors-phase} à la place.
    \item Infliger des blessures à la Horde lui fait \textbf{meuler autant de cartes}.
    \item La Horde \textbf{n'a pas de cimetière}.
    \item Les joueurs gagnent lorsqu'il n'y a \textbf{plus de carte en jeu ou dans le deck} de la Horde.
\end{itemize}


\paragraph{Note sur la difficulté}
%
Si vos parties sont trop simples, au lieu de faire meuler simplement, faites meuler jusqu'à révéler une carte non-jeton et la Horde joue cette carte gratuitement.

\subsubsection*{Tour de Jeu de la Horde}

Le tour de jeu de la Horde se déroule comme suit :

\begin{enumerate}
    \item Dégagement
    \item Entretien
    \item \textbf{Pioche} : La Horde remplace sa pioche par : « Révélez les cartes du dessus de votre bibliothèque jusqu'à révéler une carte non-jeton. Mettez en jeu tous les jetons révélés de cette manière, puis \textbf{lancez} cette carte gratuitement. »
    \item Phase principale
    \item \textbf{Attaque} : Toutes les créatures capables d'attaquer le font.
    \item Phase principale
    \item Fin de tour
\end{enumerate}

\subsubsection*{Règles Spécifiques selon le nombre de joueurs}

\begin{center}
\begin{tabular}{|c|c|c|c|c|}
    \hline
    \textbf{Nombre} & \textbf{PV de départ} & \textbf{Taille du deck Horde} & \textbf{Pioche de la Horde} & \shortstack{\textbf{Multiplicateur}\\\textbf{de jetons}} \\
    \hline
    1 joueur & 40  & 75  & 1  &  \\
    \hline
    2 joueurs & 40  & 100  & 1  &  \\
    \hline
    3 joueurs & 60  & 100  & 2  &  \\
    \hline
    4 joueurs & 80  & 200  & 2  & x2  \\
    \hline
\end{tabular}
\end{center}


\subsubsection*{Avec d'autres variantes ?}

Ce mode de jeu « Horde » est compatible avec d'autres variantes, « Plane Chase » ou « Vanguard »!
Mais c'est incompatible avec « Shogun », « Treachery » ou « Arch Enemy ».

\end{document}
