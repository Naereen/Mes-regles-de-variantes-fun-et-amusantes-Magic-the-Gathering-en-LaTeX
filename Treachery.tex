\documentclass{adsdoc}

\newcommand{\doctitle}{Variante en EDH Commander : les rôles cachés « Treachery »}

\title{
    \vspace*{-2cm}
    Variante EDH Commander : les rôles « Treachery »\\
    \Large{Ajoutez des rôles cachés à vos parties d'EDH Commander, pour plus de fun et de variété !}\\
    {
        \large
    \vspace*{0.2cm}
    -- \textbf{Nombre de joueurs :} 4 à 6 joueurs (idéalement 5) \\
    -- \textbf{Durée d'une partie :} plus rapide qu'une partie d'EDH à 5, 1H à 1H30 \\
    -- \textbf{Matériel requis :} le paquet des cartes « rôles cachés Treachery » fourni, et une carte d'explication, en plus de vos decks d'EDH préférés
    }
}
\date{}

\begin{document}

\maketitle
\thispagestyle{fancy}

\vspace*{-1.5cm}

\noindent

% \begin{center}
%     \includegraphics[height=3cm]{images/treachery-masks.png}
% \end{center}

\begin{itemize}
    \item
    La variante \textbf{Treachery} ressemble à la variante \textbf{Shogun}. Si vous n'avez jamais pratiqué cette dernière, je vous recommande de commencer par là !
    \item
    Le \textbf{Treachery} ajoute une dose de bluff et de politique à vos parties de \textbf{Magic: The Gathering} au format EDH, grâce à un rôle caché reçu en début de partie.
    \item
    Ce rôle caché spécifie votre condition de victoire.
    \item
    Cette variante s'inspire du jeu de cartes \emph{Bang!} ou de jeux de rôles cachés.
    \item
    Votre rôle caché vous offre aussi une capacité unique et très puissante, de ``dévoilement''. Vous pouvez l'utiliser une seule fois dans la partie, en payant son coût d'activation, et son effet peut retourner la partie !
\end{itemize}

\vspace*{1em}
Voici les règles du « Treachery », suivant le site \url{https://mtgtreachery.net/fr/} :


\subsection*{Distribution initiale des rôles « Treachery »}

\begin{itemize}
    \item Choisissez d'abord le niveau de votre table, puis chaque personne choisit son deck, comme dans une partie standard.
    \item
    Pour démarrer une partie de Treachery, préparez un paquet de cartes d'Identité, selon le nombre de joueurs, suivant cette répartition :
        \begin{itemize}
            \item Si vous êtes quatre, distribuez : un Seigneur, deux Assassins, un Traître (pas de Gardienne).
            \item \textbf{Si vous êtes cinq, c'est idéal !} Distribuez : un Seigneur, une Gardienne, deux Assassins, un Traître.
            \item Si vous êtes six (déconseillé, c'est dur et trop long), distribuez : un Seigneur, une Gardienne, trois Assassins, un Traître.
        \end{itemize}

    \item
    Mélangez ensemble les cartes d'Identité sélectionnées et distribuez-en une, face cachée, à chaque joueur. Les joueurs consultent secrètement leur Identité.
    \item
    Le Seigneur se révèle en retournant sa carte face visible.
    Certaines Identités de Seigneurs commencent la partie, mais pas toutes !
    \item
    Tous les autres joueurs gardent leur rôle secret.
    \item
    Que le plus diplomatique, la plus maligne ou le plus filou gagne !
\end{itemize}


\subsection*{Les règles précises et détaillées (si besoin)}

Le site officiel propose une section rédigée précisément (en anglais), dans le style des « Comprehensive Rules » officielles :
\url{https://mtgtreachery.net/rules/cr/}


\newpage

\subsection*{Règles des rôles « Treachery »}

\begin{itemize}
    \item \textbf{Rôle du Seigneur} (1 max) : votre objectif est d'être le dernier joueur en vie. Vous êtes connu de tous les joueurs. Vous commencez à 50 PV et vous commencez la partie.
    \item \textbf{Rôle de la Gardienne} (1 max) : votre objectif est de protéger le Seigneur. Si le Seigneur meurt, vous perdez également. Vous n'êtes initialement connue de personne. Si ou lorsque vous choisissez de vous révéler (cf. plus bas), vous et le Seigneur devenez alliés (vous n'êtes plus des ``adversaires'' pour tous les sorts, capacités et effets qui s'intéressent uniquement aux adversaires et non à tous les joueurs).
    \item \textbf{Rôle des Assassins} (3 max) : votre objectif est de tuer le Seigneur. Vous n'êtes initialement connu de personne. Vous pouvez vous révéler (cf. plus bas), et si des Assassins se sont révélés, ils ne comptent plus comme des Adversaires. Si le Seigneur meurt, vous gagnez la partie (les deux ou trois Assassins peuvent gagner mutuellement).
    \item \textbf{Rôle du Traître} (1 max) : votre objectif est de tuer le Seigneur, sans que les Assassins gagnent, donc après les avoir éliminés. Vous n'êtes initialement connu de personne. Si le Seigneur perd sans que les Assassins soient encore en vie, vous gagnez la partie. C'est évidemment le rôle le plus difficile.
\end{itemize}


\subsection*{Un exemple de chaque rôle possible}

\begin{center}
    \includegraphics[height=5cm]{images/062 - Leader - The Void Tyrant.jpg}
    %
    \includegraphics[height=5cm]{images/003 - Guardian - The Bodyguard.jpg}
    %
    \includegraphics[height=5cm]{images/047 - Assassin - The Sorceress.jpg}
    %
    \includegraphics[height=5cm]{images/029 - Traitor - The Time Bender.jpg}
\end{center}


\subsection*{Les cartes de rôle « Treachery » : la carte d'Identité}

Le site officiel du projet Treachery, \url{https://mtgtreachery.net/}, propose de télécharger une collection de cartes de rôle, prêtes à être imprimées et découpées comme des proxies (ou vous pouvez les commander sur un site spécialisé, comme \href{https://www.makeplayingcards.com/}{MakePlayingCards.com}).

Une carte d'Identité est une carte Magic non-traditionnelle personnalisée. Elle reste dans la zone de commandement tout au long de la partie. Son sous-type définit votre rôle, vos conditions de victoire et vos potentiels alliés.

Excepté pour le Seigneur, toutes les Identités sont généralement secrètes (face cachée), mais elles possèdent une capacité de dévoilement (« Unveil » en anglais), qui leur permet d'être tournées face visible (cela fonctionne comme la Mue : on ne peut pas y répondre), déclenchant du même coup un puissant effet.
Certaines de ces capacités coûtent très cher en mana, mais peuvent retourner la partie (ex. prendre un tour supplémentaire, contrôler le tour d'un adversaire, raser le champ de bataille, etc).

Les joueurs sont considérés comme adversaires les uns et les autres, tant que leurs Identités sont face cachée.


% \subsection*{Avec d'autres variantes ?}

% Ce mode de jeu « Treachery » est compatible avec d'autres variantes, comme « Vanguard » ou « Planechase » !
% Mais incompatible avec « Shogun ».

% Nous avons déjà testé une partie à quatre joueurs avec « Shogun », « Planechase » \textbf{et} « Shogun », en même temps.
% Cela marche très bien, il faut juste être bien concentré !

\end{document}
