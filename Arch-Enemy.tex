\documentclass{adsdoc}

\newcommand{\doctitle}{Variante en EDH Commander : « Arch Enemy »}

\title{
    \vspace*{-2cm}
    Variante EDH Commander : le « Arch Enemy »\\
    \Large{Affrontez un ennemi surpuissant ensemble}\\
    {
        \large
    \vspace*{0.2cm}
    -- \textbf{Nombre de joueurs :} 3 à 5 joueurs (idéalement 4) \\
    -- \textbf{Durée d'une partie :} généralement plus long qu'une partie d'EDH standard (1H30 à 2H) \\
    -- \textbf{Matériel requis :} un petit paquet de cartes « Arch Enemy » fourni, en plus de vos decks d'EDH préférés
    }
}
\date{}

\begin{document}

\maketitle
\thispagestyle{fancy}

\vspace*{-1.5cm}

\noindent

La variante \textbf{Arch Enemy} est conçue pour transformer une partie de \textbf{Magic: The Gathering} au format EDH en une partie « plusieurs contre un ».
Un joueur particulier reçoit des pouvoirs surpuissants et uniques, il devient alors l'« Ennemi Juré » pour cette partie, et les autres joueurs forment une équipe (quasiment comme au Troll-à-deux-têtes).

Le rôle de cette équipe est simple : surmonter ce défi et survivre à cet ennemi juré.
Il aura un total de point de vie supérieur, et un deck de cartes non-traditionnelles (des « machinations » ou \emph{schemes} en anglais), pour l'aider à affronter l'équipe des autres joueurs.

Ces cartes de machinations peuvent vraiment être très puissantes, en voici deux exemples :

\begin{center}
    \includegraphics[width=0.42\linewidth]{images/dsc-356-running-is-useless.jpg}
    \includegraphics[width=0.42\linewidth]{images/dsc-341-my-champion-stands-supreme.jpg}
\end{center}

Ce format est présent dans les règles officielles de Magic (§ 904. Arch Enemy), mais reste assez peu connu !
Si besoin (en anglais) : \url{mtg.wiki/page/Archenemy_(format)}

Voici les règles de la variante « Arch Enemy ».

\newpage

% TODO:
\subsection*{Distribution des rôles et préparation}

\begin{itemize}
    \item Choisissez d'abord le niveau de votre table, puis chaque personne choisit son deck, comme dans une partie standard.
    \item Choisissez ensuite qui sera le « Arch Enemy » pour cette partie (cela peut être le plus expérimenté, le joueur ayant un deck typique de « méchant », ou par consensus).
    \item Le « Arch Enemy » et l'équipe des héros commencent chacun la partie avec \textbf{20 points de vie} par joueurs dans l'équipe des héros. Typiquement, 60 PV s'il y a trois joueurs héros et un ennemi juré (§904.13b).
    \item Mélangez le deck de cartes « Arch Enemy », et placez-le face cachée, près du joueur désigné comme « Arch Enemy ».
\end{itemize}


\subsection*{Explications et résumé des règles}

Un joueur, désigné comme l'ennemi juré, affronte une équipe d'adversaires (les héros). Pour équilibrer les chances, l'ennemi juré commence la partie avec plus de points de vie, joue toujours en premier et conserve un deuxième paquet de cartes de stratagème surdimensionnées et puissantes dans la zone de commandement.
L'équipe des héros joue simultanément, comme dans le format Troll-à-deux-têtes (\emph{Two-Headed Giant}), avec un total de 20 points de vie \textbf{par héro}
N'importe quel coéquipier peut bloquer un attaquant déclaré par l'ennemi juré, quel que soit le joueur attaqué.

\textbf{Condition de victoire ?}
%
L'ennemi juré gagne si tous les autres joueurs sont éliminés du jeu. Tous les joueurs de l'équipe des héros, même ceux qui ont déjà quitté le jeu, gagnent si l'ennemi juré perd.

\textbf{Les machinations de l'ennemi juré (\emph{schemes})}
%
Au début de chacune des phases principales pré-combat de l'ennemi juré, celui-ci pioche la carte du dessus de son deck de machinations, la laisse face visible dans la zone de commandement et « la met en œuvre ».

\begin{itemize}
    \item
    Les machinations, comme les emblèmes, ne sont pas des permanents.
    \item
    La plupart des machinations ont une capacité déclenchée qui s'active immédiatement lorsqu'ils sont mis en œuvre.
    \item
    Cependant, les machinations avec le supertype « continu » (\emph{ongoing}) ont une capacité statique supplémentaire, ainsi que des capacités déclenchées qui les font « abandonner ».
    \item
    Lorsqu'un stratagème est « terminé » (résolu ou abandonné), il est placé au bas du paquet de machinations. Le paquet de machinations n'est pas mélangé après le début de la partie.
\end{itemize}

\begin{center}
    \includegraphics[height=3cm]{images/Archenemy_back.jpg}
\end{center}

\subsection*{Avec d'autres variantes ?}

Ce mode de jeu « Arch Enemy » est compatible avec d'autres variantes, « Plane Chase » ou « Vanguard » !
Mais c'est incompatible avec « Shogun » ou « Treachery ».

\end{document}
